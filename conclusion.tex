\Chapter{Conclusion and Future Direction}

\section{Conclusion}

The College of Extended Learning Online Registration System facilitates the processing of registrations for the College of Extended Learning (CEL). Based upon the feedback received from CEL staff, and new regulation requirements from the CSU administration, this version of the system represents an improvement over the previous version in many aspects.

First, the course manager has more control of the CELORS. For instance, when the Osher membership fee or the maximum number of courses allowed to take per Osher quarter changes, the course manager has full control to change the number as desired and it is affected immediately. Currently, the course manager must contact CEL's contract programmer to modify source codes and which usually takes several hours to reveal. The clone function makes populating courses descriptions into a new quarter catalog a more expedient process.

Second, this version of the CELORS makes the payment process more accurate, efficient, and secure. In older verisons, CEL staff extract encrypted payment information from the system and stored into a portable device, usually a floppy disk or flash drive. They bring it to an isolated computer to decrypt and print registrations out. They then enter credit card information and fees one-by-one into the credit card machine. A typo could cause the payment to not go through or charge the wrong person. With automating the payment process through Paypal, CEL staff still has control of the payments but less chance for human errors. The registrations won't go through the CELORS if students make any mistakes during the payment process.

Finally, by using the newer technologies of Spring Framework and Hibernate provided, the system runs more efficiently and is easier to maintain and implement.

\section{Future Direction}

The development of the project progressed much more slowly than anticipated. The project itself has a great deal of life ahead of it though, which can be viewed as a positive aspect, especially if future programmers take interest enough to complete the remaining tasks.

A short list of work that could be done in the future are:
\begin{itemize}
    \item{Implementing marketing analysis reports functionalities}
    \item{Adding the Paypal refund functionality}
    \item{Adding a pay-through-Paypal-page function (redirect registrating students to the CEL-customized Paypal page)}
\end{itemize}
